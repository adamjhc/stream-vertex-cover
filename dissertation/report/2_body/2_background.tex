\section{Background}

For the reader unfamiliar with concepts covered in this paper, an
explanation has been provided for each. For those simply unfamiliar with
abbreviations, a glossary has been provided.

\textbf{Graph Theory}: The study of mathematical structures (graphs)
which are used to show pairwise relations between objects.

\textbf{Vertex Cover}: A set of vertices such that each edge of a graph
is incident to at least one vertex of the set. The problem of finding a
minimum vertex cover (the smallest possible) is a classical optimization
problem and is a typical example of an NP-hard problem. The decision
version (where we only want a yes/no answer) is known as the Vertex
Cover Problem. Formally, given a graph \(G = (V, E)\) and a vertex cover
\(V'\): \[
    V' \subset V \text{ such that } \forall (u, v) \in E \Rightarrow u \in V' \vee v \in V'
\] \textbf{Maximal Matching}:

\textbf{Parameterized complexity}: A branch of computational complexity
theory that focusses on classifying computational problems according to
their inherent difficulty with respect to multiple parameters of the
input or output. The complexity of the problem is then measured as a
function of those parameters. The vertex cover problem is
fixed-parameter tractable, meaning that, while it may be NP-complete in
terms of the input size only, it is polynomial in the output of a vertex
cover size \(k\).

\textbf{Fixed-Parameter Tractable} (FPT): A subset of parameterized
problems, those that can be solved by algorithms that are exponential
only in the size of the parameter but polynomial in the size of the
input. These algorithms allow for efficient solving for small values of
the fixed parameter.

\textbf{Parameterized Vertex Cover}: Also known as k-VC, the vertex
cover problem is posed as a decision problem in which we are given a
graph \(G\) and a positive integer \(k\) and we must find out whether
\(G\) has a vertex cover of size at most \(k\). The \(k\) value can be
thought of as a ``budget'' to spend on the vertex cover. If we are
limited but such a budget then we have no reason to consider solutions
that exceeds this.

\textbf{Streaming Algorithm}: An algorithm designed for processing
either a bounded or unbounded data stream. Bounded streams may be
replayed with either a fixed or random order. Unbounded streams are
typically used for aggregation of data.

\textbf{Streaming Model}: The stream is split into blocks of data

\textbf{Kernelization}: Kernelization is a pre-processing method for
minimising datasets into their core components known as a kernel.
Processing completed on such a kernel will return the same output as
that would be returned had the processing been run on the entire
dataset.

\textbf{Branching}: Trees have been used as an abstract data type in
computer science for decades. They provide relatively easy logarithmic
complexity, due to the fact that they split their data into \(n\)
sections recursively, and are simple to understand and implement,
leading to them being a core concept in any University
introduction-level algorithms course.