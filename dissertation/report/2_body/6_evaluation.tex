\section{Evaluation}

Overall, we consider our work a success.

\subsection{Visualisation}

We were able to create two programs that show how each algorithm works
step by step. Visually, we find them interesting because it's always
nice to see things being built up piece by piece.

add some nicer bits before tearing myself apart

As for the educational factor, whether they act as adequate aids in
learning how each algorithm works, remains to be seen. If a student were
to look at the visualisation before knowing how the algorithm behind
worked then they would most likely struggle to gain any information from
it. If used as a part of the learning process, we believe they could be
invaluable in connecting the theory with students' representation they
have constructed in their minds. This could be improved upon to take up
a larger portion of the learning process. Using NetworkX and Matplotlib
allowed for convenient code copied over from implementations but lacked
features such as interactivity and a storyline for students to follow to
guide their understanding of each algorithm. Command-line programs, even
with extensive documentation, are inherently less user-friendly than,
say, a web page. Considering this, if we were to take a second shot at
this task, we would probably attempt to create a web-based educational
experience. A JavaScript library such as D3.js \cite{bostock2011d3}
would give the flexibility to both accurately visualise the algorithm
and allow the user to play around with values and the graph live in
their browser. The page would be able to lead the student through how
the algorithm works step by step and finish with the visualisation.

\subsection{Performance Benchmarking}

\begin{itemize}
    \item
          allow more time to get better results
    \item
          expand testing to graphs from different domains
\end{itemize}

\subsection{Stream Implementation}

The stream implementation served as a great introduction for us into the
world of streaming frameworks, so, for that alone, we think it served
its purpose. As a system, we believe the architecture is cleanly built
and can be used as a starting point of anyone building a stream
processor using Kafka.

The results shown on the front end are presented in a table drawn using
box-drawing characters. This is passed as a string from the server where
it is generated. It was done this way for the sake of simplicity however
doesn't allow for much flexibility. Along with how the results are
displayed, the results themselves are lacking in information. In a
system where jobs could be running for days at a time, it would be
useful to have progress logging as well as, on completion, details
relating to the job itself. This could include things like runtime and
memory usage. The system as a whole could be expanded in terms of job
management, possibly allowing for pausing of jobs, jobs running in
parallel, and queuing jobs. But that is beyond the scope of this
project.

As mentioned before, the Producer was built purely for development
purposes and we wouldn't expect anyone to follow in these steps for a
production environment. It was limited in the fact that the graphs it
was able to provide was hard-coded. Further work on this project would involve expanding the Producer's capabilities to be more dynamic as well as testing different data sources as the Producer to see how it would perform.

As shown in the sequence diagrams, the difference in control flow
between the kernelization and branching algorithms is a sign of
inflexibility so any attempt at expansion on the number of algorithms
the system provides access to will be met with resistance. This could be
solved be designing an API for the Producer. This would include most of
the common graph algorithm requirements. Having control over this design
could allow for some algorithmic alterations. Take, for example, the
branching algorithm, a depth-first search. In traditional
implementations, it is not necessary to have to start from the root node
after each path traversal. At the cost of memory, the stream algorithm
could be changed so that a trail of breadcrumbs (states of the vertex
cover) could be left behind and picked up after finishing a path rather
than restarting. This would only be possible if the Producer had the
functionality to be able to serve a graph from a specific starting point
(remember the edges are always the same order). These kinds of
alterations would each have their own time and place for usage but
having the ability to implement something like this gives more tools for
algorithm designers to use which is never a bad thing.