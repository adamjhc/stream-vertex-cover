\section{Introduction}

\subsection{Vertex Cover}

Calculating the vertex cover of a graph is a classical graph-theoretic problem.
Posed as a decision problem, called the vertex cover problem (or \(k\)-Vertex
Cover), we have to find whether a vertex cover of maximum size \(k\) exists for
a given graph. This problem was one of Karp's 21 NP-complete problems
\cite{karp1972reducibility}.

Imagine a heavily connected road network in a city. The city wants to figure
out the most cost-effective placement of cameras so that they are able to see
every road (we assume the cameras can see 360\(^\circ\)). The way of
calculating this mathematically would be as a vertex cover of a graph where
each intersection is a node and each road between them is an edge. For modern
cities, this graph can be too big to compute using traditional methods. So we
need updated methods to handle this. The city now realises that they only have
a certain number of cameras they're able to put up, \(k\) cameras, and so
doesn't care for any solution that exceeds their budget. The city also decides
the project won't be worth the investment if they're not able to cover the
entire city. This is now the vertex cover problem.

Graph Theory as a field of study is about understanding these abstract ideas
and then being able to apply them to real-life scenarios. A graph is a model
for objects and the relationships between them. This has many more applications
than city road networks; uses have been found in many other physical and social
sciences, from brain imaging \cite{vecchio2016brain} to social network analysis
\cite{grandjean2016social}

\subsection{Parameterized Complexity}

The theory of parameterized (time) complexity was formalised by Downey and
Fellows \cite{downey1999parameterized} as a workaround to NP-complete problems.
This quote from the book sums it up quite nicely:

\begin{quote}
    ``\dots the deal we attempt to cut with the devil is one that confines the
    "inevitable" combinatorial explosion (which we \textit{expect} for
    interesting problems) to some limited aspect (or distribution of the
    problem).''
\end{quote}

That is, a practical algorithm can be found for a problem of any input size as
long a parameter \(k\) is kept small. Problems that are able to solved in this
way are known to be fixed-parameter tractable (FPT). Vertex Cover is one such
problem.

Most of this research has been focused on time complexity, which has allowed us
to run typically intractable problems in a way that limits their runtime to
make them tractable. Now research has started on parameterized memory
complexity. This aids us in running algorithms with typically intractable
memory usage, no matter the runtime. Introducing a parameter by which to limit
their memory usage allows us to run algorithms based on our maximum memory
budget and thus run them at the limits of what is possible on our current
machines.

\subsection{Big Data and Streaming}

Big Data is the field of processing datasets too large for traditional methods.
In 2014, the UK Government identified Big Data as one of ``eight great
technologies which will propel the UK to future growth''
\cite{intellectualpropertyoffice2014}. Research into it is therefore necessary
to facilitate this.

These datasets have become larger than what we can store on hard drives. The
solution is to not store the dataset: stream it one item at a time. Streaming
algorithms have been developed to handle this, being able to gather information
while having access to a limited amount of memory.

Streaming is now widely used in industries for processing real-time data such
as sensor data for use in monitoring. Streaming is also used in many systems
for logging purposes. Actions that users make in a system need to be tracked
for analytics so whenever they make an action an event log is created and sent
to a stream of all the other users' actions. These can be then aggregated very
quickly and efficiently to gain insight from.

\subsection{Aims}

Our work here focuses on two algorithms developed by Chitnis et al: a branching
method for solving \(k\)-VC \cite{chitnis2019towards} and a kernelization
method \cite{chitnis2015parameterized}. These algorithms are designed for
non-dynamic undirected graph streams. Non-dynamic means that the only messages
received from the stream will be edge insertions as opposed to a dynamic stream
which would also include edge deletions.

Our main aim is to create a foundation for further work to be built upon. We
identify three areas which would increase accessibility into the field.

\begin{enumerate}
    \item
          As with any widely adopted algorithm, a clear explanation of how the
          algorithm works in practice is essential. This includes efforts to
          visualise the algorithms in action. As with most state-of-the-art
          research it takes many years of studying to even be able to
          understand what the papers are talking about, let alone the
          pseudocode that is provided.
    \item
          Practical evidence of the performance of the algorithms. While Big-O
          notation is a good indicator of runtime and memory performance, those
          only refer to average performance which few, if any, datasets will
          fit. Seeing whether your problem area is suitable to have a specific
          algorithm applied to it is crucial.
    \item
          Aid in choosing the correct tools. There exist many tools and
          platforms nowadays for the development of streaming systems. As with
          many modern tooling, each have their fair share of buzzwords and
          jargon users are forced to wade through before fully understanding
          what a tool's function is.
\end{enumerate}

\subsection{Motivation}

In the past, streaming algorithms had always interested us, but we lacked a
specific area to be able to begin learning as the topic is very broad. This is
when the area of graph theory was brought to us by our supervisor. Dr. Chitnis
had previously been researching the problem of parameterized vertex cover. We
found that none of the algorithms he covered had ever been put into practice,
only ever written theoretically. This project gave us the platform to be able
to spend time learning about streaming algorithms and frameworks.
